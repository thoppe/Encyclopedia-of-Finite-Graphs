\documentclass[12pt]{article}
\usepackage{amsmath}
\usepackage{amsfonts}
\usepackage{hyperref}

\newcommand{\OEIS}[1]
{\href{https://oeis.org/#1}{OEIS \texttt{#1}}}


\begin{document}

\title{Integer sequence discovery from small graphs}
\author{Travis Hoppe and Anna Petrone}
\date{\today}
\maketitle

\begin{abstract}
We exhaustively enumerate all simple, connected graphs and compute a selection of invariants. 
Invariants sequences were checked against OEIS and updated/added if the database is incomplete.
We combined sequences to make new sequences and use this suggest relationships among the invariants.
We showed that we can readily visualize any sequence of graphs with a given criteria.
We released the code in a fully open-sourced manner allowing others to build off this work.
The work can be extended by considering other graph classes like trees or regular graphs.
\end{abstract}

\section{Introduction}

An invariant is a property of a graph that is preserved under isomorphism. 
Invariants can be simple binary properties (is the graph a tree?), integers (chromatic number), polynomials (chromatic or Tutte polynomials), sets (dominion sets) or even graphs themselves (subgraph and minor matching).
In this paper, we restrict our scope to invariant integers.

We are primarily concerned with the sequences produced by graph invariants, the combinatorial problem of how many graphs of a given class satisfy a given criteria.
Let a graph be defined as the pair $G = (V,E)$, where $V$ is a set of vertices and $E$ is a set of edges with. 
Define a class of graphs $\mathcal{C}$ as a set of isomorphically distinct graphs that satisfy a given criteria.
Group the graphs into subsets such that
\begin{equation}
\mathcal{C} = \mathcal{C}_1 \cup \mathcal{C}_2 \cup \mathcal{C}_3 \cup \ldots
\end{equation}
and the set $\mathcal{C}_n$ contains only graphs of order $n$.
Define an \textit{invariant graph sequence} as the sequence defined by the pair $S = (f, \mathcal{C})$ and 
\begin{align}
S(f, \mathcal{C}) 
&= |f(\mathcal{C}_1)|, |f(\mathcal{C}_2)|, |f(\mathcal{C}_3)|, \ldots  \\
&= s_1, s_2, s_3, \ldots
\end{align}
%
Here, $f$ is an invariant equality that further subdivides each set $\mathcal{C}_n$. 
For example, if $f_\text{tree}$ is the indicator function that determines if the graph is a tree of not and $\mathcal{C}_\text{simple}$ contains all simple connected graphs then the sequence generated is \OEIS{A000055},
\begin{align}
S(f_\text{tree}, \mathcal{C}_\text{simple}) = 1, 1, 1, 1, 2, 3, 6, 11, 23, \ldots
\end{align}

In this paper, we restrict the classes examined to $\mathcal{C}_\text{simple}$.
Provided one had a means of enumeration, extension to other classes would be straightforward.

\section{Methods}
Using the \texttt{geng} from \texttt{nauty}, we first enumerated all connected simple graphs of order $N \le 10$.

\section{Discussion}
\section{References}

\end{document}
