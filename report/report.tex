\documentclass[12pt]{article}
\usepackage{amsmath}
\usepackage{amsfonts}
\usepackage{hyperref}

\newcommand{\OEIS}[1]
{\href{https://oeis.org/#1}{OEIS \texttt{#1}}}

\newcommand{\SEQ}{\mathcal{S}}
\newcommand{\CLASS}{\mathcal{C}}
\newcommand{\SIMPLECLASS}{\mathcal{C}_\text{simple}}

\newcommand{\ie}[0]{i.e.\ }


\begin{document}

\title{Integer sequence discovery from small graphs}
\author{Travis Hoppe and Anna Petrone}
\date{\today}
\maketitle

\begin{abstract}
We exhaustively enumerate all simple, connected graphs and compute a selection of invariants. 
Invariants sequences were checked against OEIS and updated/added if the database is incomplete.
We combined sequences to make new sequences and use this suggest relationships among the invariants.
We showed that we can readily visualize any sequence of graphs with a given criteria.
We released the code in a fully open-sourced manner allowing others to build off this work.
The work can be extended by considering other graph classes like trees or regular graphs.
\end{abstract}

\section{Introduction}

Talk about OEIS (ref).

An invariant is a property of a graph that is preserved under isomorphism. 
Invariants can be simple binary properties (is the graph a tree?), integers (chromatic number), polynomials (chromatic or Tutte polynomials), sets (dominion sets) or even graphs themselves (subgraph and minor matching).

We are primarily concerned with the sequences produced by graph invariants, \ie the combinatorial problem of how many graphs of a given class satisfy a particular criteria.
Let a graph be defined as the pair $G = (V,E)$, where $V$ is a set of vertices and $E$ a set of edges. 
Define $\CLASS$ as a class of graphs which form an isomorphically distinct set of graphs that satisfy a given criteria.
Group the graphs into subsets such that
\begin{equation}
\CLASS = \CLASS_1 \cup \CLASS_2 \cup \CLASS_3 \cup \ldots
\end{equation}
where $\CLASS_n$ contains only graphs of order $n$.
From here, define a sequence of subsets of the graph class
%
\begin{align}
\SEQ(f, \CLASS) 
&= f(\CLASS_1), f(\CLASS_2), f(\CLASS_3), \ldots 
\end{align}
%
where $f$ is some invariant equality that further subdivides each set $\CLASS_n$. 
Let 
$S(f, \CLASS) = |f(\CLASS_1)|, |f(\CLASS_2)|, |f(\CLASS_3)|, \ldots$
be the sequence of integers defined by $\SEQ(f, \CLASS)$.
For example, if $f_\text{tree}$ is the $\{0,1\}$ indicator function that determines if the graph is a tree, and $\SIMPLECLASS$ is the set of all simple unlabeled connected graphs
%
\begin{align}
S(f_\text{tree}, \SIMPLECLASS) = 1, 1, 1, 1, 2, 3, 6, 11, 23, \ldots
\end{align}
%
this sequence is \OEIS{A000055}.

Two sequences of the same class are subsets of each other $\SEQ_a(f,\CLASS) \subseteq \SEQ_b(g, \CLASS)$, if $f(\CLASS_i) \subseteq g(\CLASS_i)$ for all $i\ge0$.
Equality of two sequences $\SEQ_a = \SEQ_b$, implies $\SEQ_a \subseteq \SEQ_b$ and $\SEQ_b \subseteq \SEQ_a$. 

Similar to the \textit{House of Graphs} project (ref), we say that a relation between two invariants is \textit{suggestive} $\SEQ_a \subseteq_k \SEQ_b$ if $f(\CLASS_i) \subseteq g(\CLASS_i)$ for $0 \le i \le k$.
Two invariant equalites are \textit{exclusive} if $\SEQ_a \cup_k \SEQ_b$ if $f(\CLASS_i) \cup g(\CLASS_i) = \emptyset$ for $0 \le i \le k$.
In addition to contributing to the OEIS, a secondary goal is to identify all \textit{suggestive} and \textit{exclusive} relations between the invariants studied for $k=10$.

In this paper, we restrict the classes examined to $\SIMPLECLASS$.
Since it is implied that all sequences in this paper refer to simple graphs, we let $S(f,\SIMPLECLASS)=S(f)$ for convenience.
Provided one had a means of enumeration, an extension to other classes would be straightforward.


\section{Methods}
Using the \texttt{geng -c} from \texttt{nauty} (ref), we enumerated the class of $\SIMPLECLASS$ up to order $N \le 10$.
For each graph we computed a series of invariants, which for completeness, are described below.

\subsection{Invariants}

The \textit{automorphism number} is the cardinality of the automorphism group of the graph, where each member is a mapping of the vertices onto themselves which preserve isomorphism. We use BLISS (ref) to compute this.

Let $D$ denote the graph distance matrix, where $D_{ij}$ is the shortest path (geodesic) from vertex $i$ to $j$.
In $\SIMPLECLASS$ all distances are finite.
The \textit{eccentricity} of a vertex $k$, $\epsilon(k)$ is the max value of $D_{k j}$.
The \textit{radius}/\textit{diameter} is the minimum/maximum value of eccentricity for all vertices.

The adjacency matrix $A$ has the value at $A_{ij}$ equal to the number of edges joining vertices $i$ and $j$.
For $\SIMPLECLASS$, this is a $\{0,1\}$ matrix with zeros down the diagonal.
The \textit{spectrum}, $\lambda_1, \lambda_2, \ldots$ of a graph are the eigenvalues of $A$.
A graph is $integral$ if all the values of the graph spectrum are integral, $\lambda_k \in \mathbb{N}$.
A graph is $real$ if $\lambda_k \in \mathbb{R}$.

A graph is \textit{Hamiltonian}/\textit{Eulerian} if there is a circuit that visits each vertex/edge exactly once time.

\section{Discussion}


\section{References}

\end{document}
